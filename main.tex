\documentclass[a4paper, 12pt]{article}
\usepackage[utf8]{inputenc}
\usepackage{packages}
\begingroup

\author { Frederik Olsen \\
        \small\textit{University of Copenhagen} \\
        \small Department of Economics
\and 
        Jørgen Baun Høst\\
        \small \textit{University of Copenhagen} \\
        \small Department of Economics}


\title{Demographics and the natural rate of interest in an OLG model\thanks{Special thanks to: \textbf{Emiliano Santoro}, professor MSO at the University of Copenhagen, for supervising and giving valuable feedback. \textbf{Sune Caspersen}, analyst at The Economic Council of the Labour Movement, for helping with data collection. \textbf{Marcin Bielecki}, University of Warsaw and Narodowy Bank Polski, for making codes available and answering questions regarding the model.}\\
\vspace{0.30cm}
\Large A case study of Denmark
\vspace{0.25cm}
}

\date{April 2021}%\thanks{Revision, July 2021: Slight adjustments following comments post-deadline.}}
\setcounter{tocdepth}{2}
\begin{document}
\begin{spacing}{1}
\maketitle
\end{spacing}
%\vspace*{-.5em}
\begin{abstract}
To explain some of the current low interest rate environment, we analyze how the demographic transition has affected the natural rate of interest (NRI) in Denmark. Building on work by \textcite{bielecki2020demographics}, we do this by calibrating an overlapping generations model of an open economy with a realistic pension system. We pair our simulations with Danish micro-level data and find that the ageing of the Danish population can explain approximately half of the decline in the NRI from 1985-2020. In response to continued demographic pressure, two possible reforms to the public pension system ensure fiscal sustainability, but only an increase to the retirement age can stop the decline in the NRI. On the contrary, a decrease in pension benefits will drive the NRI further down.

\medskip
\noindent
\textbf{Keywords:} Natural rate of interest, demographics, pension systems, monetary policy, secular stagnation

\medskip
\noindent
\textbf{JEL classification:} E17, E21, E43, E52, J11
\end{abstract}


\newpage 
\begingroup
\begin{spacing}{1.25}
\tableofcontents
\end{spacing}
\endgroup

\vspace{1.5cm}
\noindent
\begin{center}
\textsc{contributions:} \\ \medskip
\textbf{Frederik Olsen}: Section 2.1, 2.3, 2.5, 2.7, 3.2, 3.4, 4.1, 5.2, 6.2 \\ \medskip
\textbf{Jørgen Baun Høst}: Section 2.2, 2.4, 2.6, 3.1, 3.3, 4.2, 5.1, 6.1, 6.3 \\ \vspace{0.33cm}
Section 1 and 7 have been written together. \\

\textbf{Character Count:} 55,829 \\

\end{center}

\pagebreak
\subfile{sections/introduction.tex}



\pagebreak

\subfile{sections/model.tex} 

\pagebreak

\subfile{sections/calibrations and exogenous variables.tex} 

\pagebreak

\subfile{sections/results.tex} 

\pagebreak

\subfile{sections/Policy implications.tex} 

\pagebreak

\subfile{sections/discussion.tex} 

\pagebreak

\subfile{sections/conclusion.tex}

\pagebreak
\begin{spacing}{1.25}
\printbibliography
\end{spacing}


\pagebreak


\appendix
\appendixpage
%Avoid all sections of TOC be printed
\addtocontents{toc}{\protect\setcounter{tocdepth}{0}}

\subfile{sections/appendix.tex}



\end{document}